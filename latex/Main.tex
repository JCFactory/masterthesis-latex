\documentclass[12pt, a4paper]{scrreprt}

\renewcommand*\familydefault{\sfdefault} 
%\usepackage[T1]{fontenc}

\usepackage[english]{babel}
%\usepackage{cite}
\usepackage[utf8]{inputenc}
%\usepackage[onehalfspacing]{setspace}
\usepackage{geometry, textcomp}
\newgeometry{right=2cm,left=4cm, top=2.5cm, bottom=2.3cm, footnotesep=0.5cm}
%\usepackage[acronym]{glossaries}

\usepackage[printonlyused]{acronym}  % Abkürzungsverzeichnis [nur verwendete Abkürzugen]

%\glsenablehyper

%\makeglossaries

\usepackage{savesym}
\usepackage{amsmath,amssymb,amstext}
\savesymbol{iint}
\usepackage{txfonts}

\restoresymbol{TXF}{iint}
\usepackage[automark,headsepline,ilines,komastyle]{scrpage2}
\usepackage{blindtext}
\usepackage[euler]{textgreek}
\setlength{\parindent}{0pt}
\setlength{\headheight}{1.5\baselineskip}
\renewcommand{\baselinestretch}{1.5}

\pagestyle{scrheadings}
\clearscrheadfoot
\ihead[]{}
\chead[]{}
\ohead[]{\headmark \hfill \thepage}
\ifoot[]{}
\cfoot[]{}
\ofoot[]{}

\setheadsepline[\textwidth]{1pt}
\usepackage{tabularx}
\usepackage{colortbl}
\usepackage{multirow}
\usepackage{hhline}
\usepackage{array}
\usepackage{tocloft}
\usepackage[hidelinks]{hyperref}
\tocloftpagestyle{scrheadings}
\renewcommand{\chapterpagestyle}{scrheadings}
\usepackage[font=footnotesize]{caption}

\usepackage{tikz}
\usepackage{rotating} 

\newenvironment{packed_item}
	{\begin{itemize}
			\setlength{\itemsep}{0pt}
			\setlength{\topsep}{0pt}
			\setlength{\parsep}{0pt}
			\setlength{\parskip}{0pt}}
		{\end{itemize}}
	
\usepackage[style=authoryear, natbib=true, backend=biber]{biblatex}

\renewcommand{\nameyeardelim}{ }
\usepackage[babel,german=guillemets]{csquotes}

\makeatletter

\newrobustcmd*{\parentexttrack}[1]{%
	\begingroup
	\blx@blxinit
	\blx@setsfcodes
	\blx@bibopenparen#1\blx@bibcloseparen
	\endgroup}

\AtEveryCite{%
	\let\parentext=\parentexttrack%
	\let\bibopenparen=\bibopenbracket%
	\let\bibcloseparen=\bibclosebracket}

\makeatother

\usepackage{pstricks}
\usepackage{pstricks-add}

\bibliography{Lit.bib}

\usepackage[final]{pdfpages}

\begin{document}
		\begin{titlepage}
			\begin{center}
			%\setlength{\headheight}{1.5\baselineskip}
			\renewcommand{\baselinestretch}{1.5}
					\textbf{\large FOM - Hochschule für Oekonomie \& Management \\
						Hamburg \\
						\ \\
						Master-Studiengang Big Data \& Business Analytics \\
						5. Semester \\
						\ \\
						Development of a solution to detect skin lymphoma  \\
						and to improve treatment processes} \ \\
						\ \\
						
						
					\textrm{
						\ \\
						Betreuer: Prof. Dr. David Matusievicz \\
						\ \\
						Autor: Jacqueline Franßen \\
						\ \\
						Matrikel-Nr: 496804 \\
						\ \\
						5. Fachsemester \\
						\ \\
						Hamburg, den 28.02.2021 \\
						}
			\end{center}
		\end{titlepage}

%\includepdf{Image/Deckblatt.pdf}

			\setcounter{tocdepth}{3}
			\setcounter{secnumdepth}{3}		
			\pagenumbering{Roman}
			\thispagestyle{empty}
			\pdfbookmark{\contentsname}{toc}\tableofcontents
			\newpage
			\listoffigures
			\listoftables

			\pagenumbering{arabic}
			\thispagestyle{empty}
\chapter{Abstract}\label{abstract}

This scientific article focusses on the development of a chatbot and the analysis of blood pressure measurements.
The main purpose of this scientific article is to develop...
This can lead to...
What is more, .... 
The first business case is that...
Second,. 
Furthermore, ...
Third,...
Fourth,...
Last, ...

\chapter{Introduction}\label{introduction}

\section{Problem statement}
According to....
Another problem are 
All of these use cases can be automated and implemented by an application....

\section{Aim and scope of this work}
%describe aim
The first aim of this scientific work is to develop a solution ...
%describe scope
What is important, the developed model is only a reference model....

\chapter{Fundamentals}\label{fundamentals}

\section{Related Work}

\subsection{\ac{IARC}}

\section{Current algorithms, solutions to predict metastasis}

\section{Overview: Apps to predict metastasis}

There already exist some applications to predict certain types of cancer worldwide, such as \footnote{\cite{iarc_predict_cancer_worldwide}}. These solutions provide several information about the tumors by showing them within many different types of charts. They do not specify on a certain patient (which would also cause problems with data privacy and protection) but explain them in general.  


\footnote{\cite{vijini_gen_alg}}


\chapter{Image Recognition and App Development}
Here all used algorithms and patterns shall be explained

\section{Image Recognition}
\section{Hybride App Development}


\chapter{Analysis and Development}

\section{Design Thinking Methods}

In the following sections there are explained various kinds of design strategies to find out the real needs for the application being developed. These are adopted to two basic frameworks, the IBM Enterprise Design Thinking framework \footnote{\cite{ibm_edt}} as well as the Design Kit, proposed by IDEO.ORG \footnote{\cite{design_kit}}. Some of the strategies are modified a little bit and were implemented by a single person which can affect the objectivity and diversity of the methods. Nevertheless, they generated a highly usable product and many ideas for future development cycles. 

\subsection{Vertical latter}

\begin{figure}[h!]
	\centering
	\includegraphics[width=1\textwidth]{images/verticallatter.jpg}
	\caption{Vertical latter which defines the development goals over time}
	\label{verticallatter}
\end{figure}

In the given figure, a vertical letter shows the initially planned goals and features of the app. The X-Axis shows the time steps whereas the Y-Axis explains the complexity of tasks. The higher on task is mentioned, the more complex it is to realize.

\subsection{How might we?}

\begin{figure}[h!]
	\centering
	\includegraphics[width=1\textwidth]{images/howmightwe.jpg}
	\caption{"How might we?" - table to define the process of development}
	\label{verticallatter}
\end{figure}

\subsection{Stakeholder Map}

\begin{figure}[h!]
	\centering
	\includegraphics[width=1\textwidth]{images/stakeholdermap.jpg}
	\caption{Stakeholder Map which shows all groups of stakeholders as well as their relationships}
	\label{verticallatter}
\end{figure}

\subsection{Empathy Map}

\begin{figure}[h!]
	\centering
	\includegraphics[width=1\textwidth]{images/empathymap_radiologist.jpg}
	\caption{Empathy Map of radiologist
	\label{verticallatter}
\end{figure}

\begin{figure}[h!]
	\centering
	\includegraphics[width=1\textwidth]{images/empathymap_patient.jpg}
	\caption{Empathy Map of patient}
	\label{verticallatter}
\end{figure}

\begin{figure}[h!]
	\centering
	\includegraphics[width=1\textwidth]{images/empathymap_cfo.jpg}
	\caption{Empathy Map of \ac{cfo} of hospital}
	\label{verticallatter}
\end{figure}


\subsection{Hills}

\begin{figure}[h!]
	\centering
	\includegraphics[width=1\textwidth]{images/hills.jpg}
	\caption{"How might we?" - table to define the process of development}
	\label{verticallatter}
\end{figure}

\subsection{User Journey}

\begin{figure}[h!]
	\centering
	\includegraphics[width=1\textwidth]{images/userjourney.jpg}
	\caption{User Journey which describes the interaction of key users and app}
	\label{verticallatter}
\end{figure}

\section{User Research}
\subsection{User interviews}
\subsection{What is important? - Relevant features}
\subsection{How can processes be improved?}


\section{Development of system to predict metastasis and recidives} 

\section{Data Research}
\subsection{Kaggle dataset}
\section{\ac{CRISP-DM}: Model Planning and Learning}
\section{Testing}
\section{Results} 

\chapter{Results}
\section{Validation of results}
\section{Limitation in the development process}

\chapter{Conclusion and Outlook}
\section{Conclusion}
\section{Outlook}


\chapter{Abbreviations}
\begin{acronym}[CRISP-DM]
\acro{crisp-dm}[CRISP-DM]{CRoss-Industry Standard Process for Data Mining}
\acro{cfo}[CFO]{Chief Financial Officer}
\end{acronym}

\printbibliography[heading=bibintoc]

\chapter{Appendix A}\label{appendix a}
\input{Anhang}
\chapter{Appendix B}\label{appendix b}
%\includepdf[pages=-]{keras_classifier.pdf}

\end{document}
